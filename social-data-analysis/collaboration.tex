Another key factor for innovation in the office outlined by \cite{hub13} is the hunt for and utilization of talented people. One major factor of keeping employees happy is to support different styles of work. This includes silent working areas to focus on projects, rooms to receive phone calls, but also areas where ``the atmosphere is conducive to innovation'' \cite{tieto}.

If these areas are provided, the employees must be able to freely move between these areas. One solution for this are non-fixed working desk, i.e. employees pick their working place when after they arrive in the office. \cite{occupiee}.  also outlines that flexible offices are needed as more employees work mobile and may rarely return to the office. In a flexible office environment with shared desks, the smart and interactive office must provide ways to find out where space is available and where colleagues are currently working \cite{tieto}.

Such technologies can be also be utilized to improve working together. If a project requires specialists, spontaneous meetings can be held by seeing who's currently available, where they are and what meeting room can be used \cite{tieto}. Similar results are mentioned \cite{hbcommunications} where the authors describe how streamlined communication and improved connectivity leads to better and faster collaboration between organization experts \footnote{\cite{hbcommunications} also mentions how automation of heating and lighting can lead to a more ``fun'' office improving the well-being of employees.}.

The importance of exciting workplaces for the creativity of employees is also mentioned in \cite{roomzilla3}. While typically it is assumed that such a playful environment may be detrimental to the productivity of employees, such environments can actually lead to a more creative and productive employees \cite{metroffice}. This in turn leads to better results and therefore a more successful business.

% Speaking of communication, having a smart workplace can have a big impact on overall corporate culture within an organization. People expect their workplaces to have technology that’s at least as good as what they have at home—preferably better—and this is especially true for those digital natives, the Millennials. Studies show that happy employees are productive employees and productive employees lead to prosperous businesses. Equally as important is the ability of companies to not only attract, but also to retain top talent. When you create a workplace that is a pleasure to be in and work in and provide state-of-the-art technology, it goes a long way toward making employees want to stay around. easier life for office management: lights turn on and off automatically, temperature regulation via thermostats \(\rightarrow\) safe energy. more fun workplace with: coffee pots, fitness equipment, or wearables that track fitness and reward employees for reaching certain fitness goals\cite{hbcommunications}

