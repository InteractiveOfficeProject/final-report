% \paragraph{Introduction}
% The goal of our project is to improve the office by utilizing smart and interactive technologies. 
% One area of interest is to improve the efficiency of employees and the office itself which is 
% investigated in section \ref{sec:sda-efficiency}. Better collaboration is a very important part of 
% efficiency and will be investigated in section \ref{sec:sda-collaboration}. Section \ref{sec:sda-comfort} 
% will take a at how a comfortable and enjoyable working environment can increase the productivity and 
% creativity of workers. Finally, \ref{sec:sda-safety} reports shortly on how modern technologies can 
% be make the working environment safer.

\paragraph{Related Work}
Studies in real working environments are hard to do: the situation cannot be controlled and the 
study must impact the working people as less as possible. To conduct research, a controlled 
environment is needed. Researchers at the university of Kaiserslautern created the living lab to 
merge these requirements. The living lab is an open space office that was designed with the goal to 
conduct research in the area of Smart Offices \cite{living-lab}.


The living lab has researched different types of technologies. Examples include electrochromatic 
glass which can be turned dark on demand to prevent sunlight from shining trough or personlaized air 
flow optimization for workers. The living lab tests these technologies in simulations and real life 
situations. Another area of research is light and acoustic optimization. Good lighting and acoustics 
cannot be measured directly as most people only notice bad lighting or bad acoustics \cite{living-lab}.


\paragraph{Social Data Analysis: Areas of Interest}
\subparagraph{Efficiency}\label{sec:sda-efficiency}
Improving the efficiency of workers is one of the key drivers of change in workplace strategies 
according to \cite{hub13}. A smart or interactive office can help with this goal. Tools and software 
can help to reduce the time employees spend on tasks that are not directly related to profitable 
goals of the company.

Both \cite{iotagenda} and \cite{roomzilla9} describe that managing rooms is a tedious task. 
Roomzilla -- a software company selling room booking software -- estimates in \cite{roomzilla9} how 
much cost managing rooms without software can cause. The authors assume that office manager managing 
room bookings spends around 90 minutes per day to do so. Based on the average salary of an office 
manager in the US (\(20,65\text{USD}\)) the final estimate of this task is around \(681,45\text{USD}\) 
per month. The authors also mention some problems that can cause hidden costs: both late running 
meetings and overbooked rooms prevent employees from using their time to do actual work. 

The authors of this article instead describe how tracking room usage can lead to a better use of 
time. They propose a system that tracks room usage and makes the data available via Outlook or an 
appropriate alternative. This also helps employees to find a room when needed and therefore leads to 
less distractions and waste of time \cite{iotagenda}.

The tracked usage patterns can be utilized to save time and also to conserve energy. A smart or 
interactive office can turn on lights and devices in advance. In turn, the lights and devices can 
also be switched off when they are not needed anymore \cite{hbcommunications}.


\cite{hbcommunications} also outlines how AI and machine learning could be used to save employee 
time. They use the example of a smart call system with an automated menu that learns to direct calls 
to the correct department. This will reduce the time that is spent on redirecting callers and 
therefore improve the employees' efficiency. However, machine learning could also be used to suggest 
the best meetings times or predict when rooms are available.

Some other causes of wasted time in companies are technical problems in \cite{roomzilla3}. 
Especially in meetings these problems can take up some time. Sub-optimal setups, used (but not 
booked) rooms, and technical failures can cause delay of the meeting start and as such potentially 
lead to long-running meetings. The authors of \cite{roomzilla3} propose streamlined processes to 
make meetings more efficient. 

\subparagraph{Collaboration}\label{sec:sda-collaboration}
Another key factor for innovation in the office outlined by \cite{hub13} is the hunt for and 
utilization of talented people. One major factor of keeping employees happy is to support different 
styles of work. This includes silent working areas to focus on projects, rooms to receive phone 
calls, but also areas where ``the atmosphere is conducive to innovation'' \cite{tieto}.

If these areas are provided, the employees must be able to freely move between these areas. One 
solution for this are non-fixed working desk, i.e. employees pick their working place when after 
they arrive in the office. \cite{occupiee}.  also outlines that flexible offices are needed as more 
employees work mobile and may rarely return to the office. In a flexible office environment with 
shared desks, the smart and interactive office must provide ways to find out where space is 
available and where colleagues are currently working \cite{tieto}.

Such technologies can be also be utilized to improve working together. If a project requires 
specialists, spontaneous meetings can be held by seeing who's currently available, where they are 
and what meeting room can be used \cite{tieto}. Similar results are mentioned \cite{hbcommunications} 
where the authors describe how streamlined communication and improved connectivity leads to better 
and faster collaboration between organization experts \footnote{\cite{hbcommunications} also 
mentions how automation of heating and lighting can lead to a more ``fun'' office improving the 
well-being of employees.}.

The importance of exciting workplaces for the creativity of employees is also mentioned in 
\cite{roomzilla3}. While typically it is assumed that such a playful environment may be detrimental 
to the productivity of employees, such environments can actually lead to a more creative and 
productive employees \cite{metroffice}. This in turn leads to better results and therefore a more 
successful business.



\subparagraph{Comfort}\label{sec:sda-comfort}
Modern LEDs provide great ways to improve the worker's productivity by adapting intensity and the 
color spectrum. It has been shown in recent research that the color spectrum directly influences the 
activity and biorhythms of people  \cite{living-lab}. \cite{iotagenda} also highlights, how smart 
lighting can be used to create a more comfortable working environment.

Another important factor of well-being is acoustics. Most open area offices are too quiet and as 
such talks between colleagues and phone calls distract other people. However, too loud environments 
are also detrimental to work. Therefore the right balance as to be found \cite{living-lab}. 

Another possibility of the smart and interactive office is the automatic regulation of room 
temperature based on the time of day. Both \cite{iotagenda} and \cite{living-lab} outline the 
importance of temperature in the well-being of employees. People expect good thermal regulation in 
the office and it is also necessary to focus on work. But not everybody does feel temperature the 
same way. The living lab therefore developed and currently a ``climatic chair'' that helps each 
individual to regulate his or her working surrounding temperature \cite{living-lab}.


\subparagraph{Safety}\label{sec:sda-safety}
Industrial workplaces like factories can be dangerous. While this is not directly related to a smart 
or interactive office, it is nonetheless an important factor in a smart workplace. \cite{sda-wired} 
lists a deadly accident which could possible have been prevented by utilizing modern technology.

In January 2012, one worker of the ArcelorMittal Burns Harbor steel-mill died while investigating 
noise in an oxygen furnace. The cause of death was a bursting pipe that released hot steam. The 
burst was caused by previously built-up pressure. A smart workplace could have prevented this 
accident by tracking pressure data in the pipe and warning workers to keep clear of the dangerous 
area.

Possible implementations of such a system could use apps, mobile devices, and wearables. In case of 
danger, acoustic and visual notifications could be send to the user. While such devices are widely 
available, \cite{sda-wired} mentions that software is lacking behind. The software of such systems 
must be intuitive to use. Also, the whole office and workplace has to be integrated: IT, machines, 
sensors, and finally the workers' devices. Since sensors produce a lot of data, \cite{sda-wired} 
also mentions that improved algorithms for streamed data analysis are needed.


% \subparagraph{Conclusion}
% The smart and interactive can help to improve the well-being and efficiency of employees in various 
% ways. One need that many of the papers outlined was the well-being of individuals as comfortable 
% employees will deliver better results.

% Collaboration and knowledge work can be improved by providing flexible working environments and 
% better ways of communication. While some tools exist in this area, there is still way for 
% improvement. 

% Another point is automation: much time is wasted by employees doing things manually that could be 
% supported well by tools, e.g. room management. Many of these tasks may be repetitive and also 
% prevent the employee from doing ``actual work'' -- which in turn provides the chance to remove or 
% simplify these tasks.
