Industrial workplaces like factories can be dangerous. While this is not directly related to a smart or interactive office, it is nonetheless an important factor in a smart workplace. \cite{sda-wired} lists a deadly accident which could possible have been prevented by utilizing modern technology.

In January 2012, one worker of the ArcelorMittal Burns Harbor steel-mill died while investigating noise in an oxygen furnace. The cause of death was a bursting pipe that released hot steam. The burst was caused by previously built-up pressure. A smart workplace could have prevented this accident by tracking pressure data in the pipe and warning workers to keep clear of the dangerous area.

Possible implementations of such a system could use apps, mobile devices, and wearables. In case of danger, acoustic and visual notifications could be send to the user. While such devices are widely available, \cite{sda-wired} mentions that software is lacking behind. The software of such systems must be intuitive to use. Also, the whole office and workplace has to be integrated: IT, machines, sensors, and finally the workers' devices. Since sensors produce a lot of data, \cite{sda-wired} also mentions that improved algorithms for streamed data analysis are needed.

% Needed: policy and protocols for data governance, privacy, data administration of personal health-information, and extensive redcordings of mic and video feeds. All of these problems are related to SW. \cite{sda-wired}

