\subsection{Efficiency}\label{sec:efficiency}

Improving the efficiency of workers is one of the key drivers of change in workplace strategies according to \cite{hub13}. A smart or interactive office can help with this goal. Tools and software can help to reduce the time employees spend on tasks that are not directly related to profitable goals of the company.

Both \cite{iotagenda} and \cite{roomzilla9} describe that managing rooms is a tedious task. Roomzilla -- a software company selling room booking software -- estimates in \cite{roomzilla9} how much cost managing rooms without software can cause. The authors assume that office manager managing room bookings spends around 90 minutes per day to do so. Based on the average salary of an office manager in the US (\(20,65\text{USD}\)) the final estimate of this task is around \(681,45\text{USD}\) per month. The authors also mention some problems that can cause hidden costs: both late running meetings and overbooked rooms prevent employees from using their time to do actual work. 

The authors of this article instead describe how tracking room usage can lead to a better use of time. They propose a system that tracks room usage and makes the data available via Outlook or an appropriate alternative. This also helps employees to find a room when needed and therefore leads to less distractions and waste of time \cite{iotagenda}.

The tracked usage patterns can be utilized to save time and also to conserve energy. A smart or interactive office can turn on lights and devices in advance. In turn, the lights and devices can also be switched off when they are not needed anymore \cite{hbcommunications}.


\cite{hbcommunications} also outlines how AI and machine learning could be used to save employee time. They use the example of a smart call system with an automated menu that learns to direct calls to the correct department. This will reduce the time that is spent on redirecting callers and therefore improve the employees' efficiency. However, machine learning could also be used to suggest the best meetings times or predict when rooms are available.

Some other causes of wasted time in companies are technical problems in \cite{roomzilla3}. Especially in meetings these problems can take up some time. Sub-optimal setups, used (but not booked) rooms, and technical failures can cause delay of the meeting start and as such potentially lead to long-running meetings. The authors of \cite{roomzilla3} propose streamlined processes to make meetings more efficient. 
