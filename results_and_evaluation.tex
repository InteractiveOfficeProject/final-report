% what are the results and outcomes of the project? 
The needfinding phase of the project identified the main stakeholders of any kind of office environments. Employees, employers and visitors are among the main actors, who regularly visit and spend considerable amount of time in the office. Furthermore, cleaners and customers visit such environment occasionally with slightly different purposes. 

% what are the identified needs? 
The literature research, interviews and observation highlighted that a connective point for all stakeholders is the fact that activities performed in the office environment are work-related. These activities typically require good atmosphere, collaboration and proper tools in order to achieve efficiency. Several tools exist already which assist office workers to perform their work more efficiently, for example by automation of repetitive tasks. 

Collaboration is often a challenge for many companies due to cultural gaps or other interpersonal issues. Office workers may not know eachother very well with respect to professional background and personal interests, which generates obstacles for fluent collaboration. Therefore, there is a possibility to enhance collaboration by enhancing relationships of co-workers.

Well-being and comfort are another aspects in establishing a good office environment, which strongly contributes to efficient work hours as well. For instance, studies show that people who take regular breaks and alter between tasks during their work are more efficient and achieve more by the end of the day. However, we concluded trough observation and literature review that some workers resist to take breaks while performing their daily work. 

% what was the idea?
Our project put focus on to develop a tool, which enhances break habits and thereby improves social relationships and well-being of individuals working in office environments. This is done by tracking worker activity by for instance pressure sensors in their chair, keyboard activity or face recognition tracking through a web camera. Ultimately, the users could have some kind of sensors that track signals of the body (for instance pulse or heart rate) to determine how "aware" the person is. The scope of such development was excluded from the project, but could be interesting for further research. 

% what was developed? 
The basis of a software were developed, which can enhance break habits of office workers. Based on the Pomodoro technique, the software notifies the person after certain amount of work time (for example 25 minutes). The software suggests employees to take a break by picking an activity/topic and by finding a partner or multiple partners to spend the break with. Activities can be chosen from a list, which is retrieved from a remote database. 

By participating in different activities with colleagues, breaks become more interactive and fun. Workers stand up from their desk, socialize, experience something new around their regular working environment. Consequently, the motivation of workers to go on breaks increases, social relationships and collaboration among colleagues enhance. 

Matching up the partners for the break is completely random at this point of time. However, it would be interesting to gather data about the background and interests of the users and perform the matching based on the interests in common. Once the break is over, users return to their desk, fill the feedback form and continue to work. The feedback form investigates the overall feeling of the person after the break is over, how he/she feels, did he/she meet somebody new at the company or did he/she learn something new about the colleagues. 

% what are the best ways to evaluate the outocomes?
Evaluation of the software was not done due to the small scope and time frame. However, we would suggest to perform a heuristic/expert evaluation on the user interface and the flow of the application. Furthermore, it would be interesting to see how users experience the software in a real environment, for example at a big corporate organization. Since the software is lacking some advanced functionality, pieces could be substituted, by for example paper based or manual work. 