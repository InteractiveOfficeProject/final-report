In the final phases of the project, a brief heuristic/expert evaluation of the software was performed by the researchers. For this evaluation, the guidelines of Nielsen and Molich were used \cite{nielsen} \cite{nielsenonline}. 

We systematically analyzed the ten points of general principles of heuristic evaluation in regard to our software. The findings are listed below. 

\begin{enumerate}
	\item \textbf{Visibility of system status}: the system status is visible in the appearing windows, however may not be clear in some cases. For example, the main window of the application does not give any instructions in what state the application is or what the user should do. In case the user is in the queue/working status, the timer on top does not clearly tell that he/she has the given amount of time left, e.g. that the timer is a countdown. The rest of the windows are otherwise explaining the status well, but there is certainly room for improvements on the main window. 
	\item \textbf{Match between system and the real world}: the terminology used by the application is understandable as the targeted users understand the relationship of work and breaks. However, there is a lack of instructions around the application's views, which could greatly contribute to the understandability of the actions a user can perform.
	\item \textbf{User control and freedom}: there is quite a few functions which the application supports at the present time and which makes the possible user actions are fairly limited. Therefore this aspect is not the most important from the list. Nevertheless, the application does not allow to change the selection of the selected activities once they are chosen, which could be a good feature to add in regard to this heuristic. 
	\item \textbf{Consistency and standards}: the appearance of the current windows is aligned with the standards of the platform on which they were taken. Accordingly, there is no major need for further improvements in this regard. However, as more features are added, this aspect should be kept in mind.
	\item \textbf{Error prevention}: error prevention was already added on the windows, where there is a possibility to make an error. For instance, users cannot join the match queue without selecting at least one activity or match up with a person, without actually selecting one. 
	\item \textbf{Recognition rather than recall}: the number of features is limited, so is the number of visual objects that are open for interaction. Users would probably not have much problems recalling what individual buttons do as their placement, labeling and appearance tells their purpose clearly. However, the lack of instructions may be an issue for new users and therefore more instructions should be added to the screens.
	\item \textbf{Flexibility and efficiency of use}: accelerators and super users were not yet considered during the project and therefore this aspect is excluded from the analysis.
	\item \textbf{Aesthetic and minimalist design}: the aesthetics of the windows could be improved by slightly bigger windows and more legible text. The application currently is a bit "boring" and is lacking of joy in use. There is not any extra or irrelevant information on the screen either, but the current design may be "too minimalistic". 
	\item \textbf{Help users recognize, diagnose, and recover from errors}: the application could guide the users with some tips and hints in case an error over just graying out the buttons on the screen. 
	\item \textbf{Help and documentation}: there is room for improvement on both in-application help and documentation for the software as these were not considered in the current state of development in any ways. 
\end{enumerate}
