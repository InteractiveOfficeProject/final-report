\subsection{Results}
% what are the results and outcomes of the project? 
The needfinding phase of the project identified the main stakeholders of any kind of office environments. Employees, employers and visitors are among the main actors, who regularly visit and spend considerable amount of time in the office. Furthermore, cleaners and customers visit such environment occasionally with slightly different purposes. 

% what are the identified needs? 
The literature research, interviews and observation highlighted that a connective point for all stakeholders is the fact that activities performed in the office environment are work-related. These activities typically require good atmosphere, collaboration and proper tools in order to achieve efficiency. Several tools exist already which assist office workers to perform their work more efficiently, for example by automation of repetitive tasks. 

Collaboration is often a challenge for many companies due to cultural gaps or other interpersonal issues. Office workers may not know each other very well with respect to professional background and personal interests, which generates obstacles for fluent collaboration. Therefore, there is a possibility to enhance collaboration by enhancing relationships of co-workers.

Well-being and comfort are another aspects in establishing a good office environment, which strongly contributes to efficient work hours as well. For instance, studies show that people who take regular breaks and alter between tasks during their work are more efficient and achieve more by the end of the day. However, we concluded trough observation and literature review that some workers resist to take breaks while performing their daily work. 

% what was the idea?
Our project put focus on to develop a tool, which enhances break habits and thereby improves social relationships and well-being of individuals working in office environments. This is done by tracking worker activity by for instance pressure sensors in their chair, keyboard activity or face recognition tracking through a web camera. Ultimately, the users could have some kind of sensors that track signals of the body (for instance pulse or heart rate) to determine how "aware" the person is. The scope of such development was excluded from the project, but could be interesting for further research. 

% what was developed? 
The basis of a software were developed, which can enhance break habits of office workers. Based on the Pomodoro technique, the software notifies the person after certain amount of work time (for example 25 minutes). The software suggests employees to take a break by picking an activity/topic and by finding a partner or multiple partners to spend the break with. Activities can be chosen from a list, which is retrieved from a remote database. 

By participating in different activities with colleagues, breaks become more interactive and fun. Workers stand up from their desk, socialize, experience something new around their regular working environment. Consequently, the motivation of workers to go on breaks increases, social relationships and collaboration among colleagues enhance. 

Matching up the partners for the break is completely random at this point of time. However, it would be interesting to gather data about the background and interests of the users and perform the matching based on the interests in common. Once the break is over, users return to their desk, fill the feedback form and continue to work. The feedback form investigates the overall feeling of the person after the break is over, how he/she feels, did he/she meet somebody new at the company or did he/she learn something new about the colleagues. 

\subsection{Evaluation}
In the final phases of the project, a brief heuristic/expert evaluation of the software was performed by the researchers. For this evaluation, the guidelines of Nielsen and Molich were used \cite{nielsen} \cite{nielsenonline}. 

We systematically analyzed the ten points of general principles of heuristic evaluation in regard to our software. The findings are listed below. 

\begin{enumerate}
	\item \textbf{Visibility of system status}: the system status is visible in the appearing windows, however may not be clear in some cases. For example, the main window of the application does not give any instructions in what state the application is or what the user should do. In case the user is in the queue/working status, the timer on top does not clearly tell that he/she has the given amount of time left, e.g. that the timer is a countdown. The rest of the windows are otherwise explaining the status well, but there is certainly room for improvements on the main window. 
	\item \textbf{Match between system and the real world}: the terminology used by the application is understandable as the targeted users understand the relationship of work and breaks. However, there is a lack of instructions around the application's views, which could greatly contribute to the understandability of the actions a user can perform.
	\item \textbf{User control and freedom}: there is quite a few functions which the application supports at the present time and which makes the possible user actions are fairly limited. Therefore this aspect is not the most important from the list. Nevertheless, the application does not allow to change the selection of the selected activities once they are chosen, which could be a good feature to add in regard to this heuristic. 
	\item \textbf{Consistency and standards}: the appearance of the current windows is aligned with the standards of the platform on which they were taken. Accordingly, there is no major need for further improvements in this regard. However, as more features are added, this aspect should be kept in mind.
	\item \textbf{Error prevention}: error prevention was already added on the windows, where there is a possibility to make an error. For instance, users cannot join the match queue without selecting at least one activity or match up with a person, without actually selecting one. 
	\item \textbf{Recognition rather than recall}: the number of features is limited, so is the number of visual objects that are open for interaction. Users would probably not have much problems recalling what individual buttons do as their placement, labeling and appearance tells their purpose clearly. However, the lack of instructions may be an issue for new users and therefore more instructions should be added to the screens.
	\item \textbf{Flexibility and efficiency of use}: accelerators and super users were not yet considered during the project and therefore this aspect is excluded from the analysis.
	\item \textbf{Aesthetic and minimalist design}: the aesthetics of the windows could be improved by slightly bigger windows and more legible text. The application currently is a bit "boring" and is lacking of joy in use. There is not any extra or irrelevant information on the screen either, but the current design may be "too minimalistic". 
	\item \textbf{Help users recognize, diagnose, and recover from errors}: the application could guide the users with some tips and hints in case an error over just graying out the buttons on the screen. 
	\item \textbf{Help and documentation}: there is room for improvement on both in-application help and documentation for the software as these were not considered in the current state of development in any ways. 
\end{enumerate}

\subsection{Future plans}
On top of the heuristic evaluation above, it would be interesting to see how users experience the software in a real environment. For example, it would be interesting direction for further research to observe how the atmosphere changes at a big corporate organization who uses this software. As the current state of the software is lacking some advanced functionality, pieces could be substituted for the study, for example by paper based or manual work. 

% what was not done and how the software could be improved?
A wide range of future development can be done on this project. Our future plans for the project include, but are not limited to 
\begin{enumerate}
	\item the detailed evaluation of the current software,
	\item experimenting with the software in a real environment,
	\item extension of the break-identification mechanism by
		\begin{enumerate}
			\item tracking keyboard activity of the user,
			\item tracking lifetime of a process on the users's computer,
			\item pressure-sensor integration to the users' chairs,
			\item analysis of body-signals to identify tiredness,
		\end{enumerate}
	\item matching algorithm for the users,
	\item automatic analysis of feedback data.
\end{enumerate}
