Since the goal of our project is to solve an existing need, a social data analysis was conducted to 
identify potential areas of improvement. Multiple blogs and other internet sources were investigated 
and compared with each other to identify common patterns and areas of interest for the smart or 
interactive office. This section will outline the four key-areas that were identified in the social 
data analysis.

\paragraph{Efficiency}\label{sec:sda-efficiency}
Improving the efficiency of workers is one of the key drivers of change in workplace strategies 
 \cite{hub13}. A smart or interactive office can help to achieve this goal by introducing tools and software to reduce the time employees spend on tasks that are not directly related to profitable 
goals of the company.

Several researchers agree that managing rooms is a tedious task \cite{iotagenda} \cite{roomzilla9}. 
Roomzilla -- a software company selling room booking software -- estimates how 
much cost managing rooms without software can cause \cite{roomzilla9}. The authors of the research assume that office manager managing room bookings spends around 90 minutes on a daily basis to perform this task. Based on the average salary of an office 
manager in the United States of America, (\(20,65\text{USD}\)) the estimation of the expenses is around \(681,45\text{USD}\) 
per month \cite{roomzilla9}. The authors also mention some problems that can cause hidden costs: both late running 
meetings and overbooked rooms prevent employees from using their time to perform actual work \cite{roomzilla9}. 

The authors of this article describe how tracking room usage can lead to a better use of 
time. They propose a system that tracks room usage and makes the data available via Outlook or an 
appropriate alternative. This also helps employees to find a room when needed and therefore leads to 
less distractions and more efficient usage of time \cite{iotagenda}.

The tracked usage patterns can be utilized to save time and also to conserve energy. A smart or 
interactive office can turn on lights and devices in advance. In turn, the lights and devices can 
also be switched off when they are not needed anymore \cite{hbcommunications}.

Other researches outlines how Artificial Intelligence and machine learning could be used to save employees' 
time \cite{hbcommunications}. In their application, a smart call system with an automated menu is introduced, that learns to forward calls 
to the correct department of the organization. This reduces the time that is spent on redirecting callers and 
therefore improve the employees' efficiency. However, machine learning could also be used to suggest 
the best meetings times or predict when rooms are available.

Some other causes of wasted time in companies cause technical problems \cite{roomzilla3}, which consume time of the workers, particularly during meetings. Sub-optimal setups, for instance used but not 
booked rooms, and technical failures can cause delay of the meeting start and as such potentially 
lead to long-running meetings, which can lead to a domino-effect. To resolve this issue, streamlined processes are suggested by researchers to make meetings more efficient \cite{roomzilla3}. 

\paragraph{Collaboration}\label{sec:sda-collaboration}
Another key factor for innovation in the office is the hunt for and 
utilization of talented people \cite{hub13}. One major factor of keeping employees happy is to support different 
styles of work. This includes silent working areas to focus on projects, rooms to receive phone 
calls, but also areas where "the atmosphere is conducive to innovation" \cite{tieto}.

If these areas are provided, the employees must be able to freely move between these areas. One 
solution for this are non-fixed working desk, i.e. employees pick their working place when after 
they arrive in the office. Researchers outline that flexible offices are needed as more 
employees work mobile and may rarely return to the office \cite{occupiee}. In a flexible office environment with 
shared desks, the smart and interactive office must provide ways to find out where space is 
available and where colleagues are currently working \cite{tieto}.

Such technologies can be also be utilized to enhance collaborative working. If a project requires 
specialists, spontaneous meetings can be held by seeing who is currently available, where they are 
and which meeting room(s) are available \cite{tieto}. Similar results are mentioned \cite{hbcommunications} 
where the authors describe how streamlined communication and improved connectivity leads to better 
and faster collaboration between organization experts \footnote{\cite{hbcommunications} also 
mentions how automation of heating and lighting can lead to a more "fun" office improving the 
well-being of employees.}.

The importance of exciting workplaces for the sake of employees' creativity is a key aspect for some offices
\cite{roomzilla3}. While typically it is assumed that such a playful environment may be detrimental 
to the productivity, such environments can actually lead to a more creative and 
productive employees \cite{metroffice}. This in turn leads to better results and therefore a more 
successful business.

\paragraph{Comfort}\label{sec:sda-comfort}
Modern LEDs provide great ways to improve the worker's productivity by adapting intensity and the 
color spectrum. It has been shown in recent research that the color spectrum directly influences the 
activity and biorhythms of people \cite{living-lab}. A recently conducted study highlights \cite{iotagenda}, how smart 
lighting can be used to create a more comfortable working environment.

Another important factor of well-being is acoustics. Most open area offices are too quiet and as 
such talks between colleagues and phone calls distract other people. However, too loud environments 
are also detrimental to work. Therefore the right balance as to be found \cite{living-lab}. 

Another possibility of the smart and interactive office is the automatic regulation of room 
temperature based on the time of day. Accordingly, the importance of temperature in the well-being of employees was already identified and highlighted by various researches \cite{iotagenda} and \cite{living-lab}. People expect good thermal regulation in 
the office and it is also necessary to focus on work. However, individuals may consider a comfortable temperature in different ways. The living lab therefore developed and currently a "climatic chair" that helps each 
individual to regulate his or her working surrounding temperature \cite{living-lab}.

\paragraph{Safety}\label{sec:sda-safety}
Industrial work environments, such as factories can be dangerous. While this is not directly related to a smart 
or interactive office, it is nonetheless an important factor in a smart workplace. 
It was studied that modern technology can be utilized to forecast and prevent deadly accidents in such environments \cite{sda-wired} .

In January 2012, one worker of the ArcelorMittal Burns Harbor steel-mill died while investigating 
noise in an oxygen furnace. The cause of death was a bursting pipe that released hot steam. The 
burst was caused by previously built-up pressure. A smart workplace could have prevented this 
accident by tracking pressure data in the pipe and warning workers to keep clear of the dangerous 
area. \cite{sda-wired} 

Possible implementations of such a system could use apps, mobile devices, and wearable devices. In case of 
danger, acoustic and visual notifications could be send to the user. While such devices are widely 
available, researchers argue that software is lacking behind \cite{sda-wired}. The software of such systems 
must be intuitive to use and the whole office ecosystem has to be integrated: IT infrastructure, machines, 
sensors, and finally the workers' devices. Since sensors produce a lot of data, 
optimized and well-constructed algorithms for streamed data analysis are needed \cite{sda-wired}.

\paragraph{Results}
The social data analysis verified, that a smart and interactive office can help to improve the 
efficiency of employees in various ways. The aspects that we derived from this data source are, as follows:

\begin{enumerate}
	\item \textbf{well-being} of individuals because comfortable employees will produce better results.
	\item \textbf{collaboration}, which can be improved by providing flexible working environments and better ways of communication. While some tools exist in this area, there is still room for improvement. 
	\item \textbf{automation}, due to the fact that significant amount of time is wasted by employees doing things manually that could be supported well by tools. Many of these tasks may be repetitive and also prevent the employee from doing "actual work" -- which in turn provides the chance to remove or simplify such tasks by automation.
\end{enumerate}
