% what kind of interviews were conducted and why? 
To gain further understanding on the needs and the opinions of the stakeholders, we decided to conduct semi-structured interviews. The reason behind choosing this method of data collection is that it provides fairly large amount of quantitative data in a short time and on low cost. We also realized that through our networks we can reach stakeholders with different background and interest and therefore easily collect data from different business areas. 

The interview questions (Appendix \ref{sec:interview-questions}) were designed together with the student tutors in one of the exercise sessions. The interviews were conducted on the upcoming weeks before the prototyping began. The transcripts of the interviews are displayed as appendices to this document. In overall five interviews were conducted with interviewees from slightly different background. During the interview phase we put some focus on the diversity of the participants so we also learn more about the problems and habits in various environments. The analysis of the findings was performed once the interviews were conducted together in a session together with the tutors.

The interviews highlighted that most of the interviewees think about themselves as they take breaks regularly. Breaks are often spent together with other colleagues, but some interviewees like to relax on their own. During the breaks interviewees do not seem to talk about work-related issues, unless necessary. This allows us to conclude that the breaks are more about relaxation, refreshing the brains and returning to work with a clear mind once they are over. 

Despite the fact that some spend breaks with colleagues, many colleagues do not know hobbies and personal background of one another. We conclude this by often hearing that colleagues tend to know what others do in their work, but not their hobbies. The reason behind this may be that private life is a personal topic and some may not like to talk about it in a working environment (as pointed out by the librarian at University of Helsinki [see transcript in the appendix]). 

Interestingly, there was an agreement among the interviewees that the good work atmosphere leads to better collaboration in the environment. This means if colleagues enjoy their time and are comfortable to work with co-workers, the interaction between them and hence the success of their collaboration increases. 