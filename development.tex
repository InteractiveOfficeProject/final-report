This section describes the development phase of our project. The development was started around the end of the needfinding phase once the team had a clear vision on the problems we were about to address and the solutions to get over those.

\subsection{Proposal}
% what do we propose? 
Analyzing the identified needs, our team had several ideas on how to overcome the challenges. We critically analyzed multiple ideas, compared them to one another, discussed the potential challenges and decided what to include in or exclude from our development project. 

We decided to focus our project towards developing a facilitator tool towards more interactive, fun and regular break habits. This proposal aims to 
\begin{itemize}
	\item be applicable in any office environment,
	\item enhance relationship between office workers,
 	\item enhance the atmosphere in the office.
\end{itemize}

By regular breaks spent with colleagues, the interaction in the environment increases, therefore people get to known, social relationships enhance. This yields in better atmosphere and therefore better productivity among colleagues. On top of that, the solution will help to avoid long, breakless work hours and therefore enhance the well being of office workers. 

We propose a software, which tracks the activity of an office worker while he/she is sitting at the desk working. The software analyzes the activity of the person and identifies a need for a break to be taken. This can be done based on time spent sitting, actively working on a task, computer keyboard activity and so on. The sitting position may be determined for instance by placing pressure sensors to the person's chair and keeping track how long pressure is pressed on the chair's plate. For the time being, the software is scaled down to use a built-in timer (also known as the Pomodoro technique \footnote{\url{https://en.wikipedia.org/wiki/Pomodoro_Technique}}) to schedule breaks. 

An extension to this data, it would be possible to track signals of the person (e.g. heart rate or pulse) to determine how "aware" he/she is and suggest the need for a break based on this data stream. This point out of the scope of this study, however would be interesting to think about in further research. 

\subsection{Project planning and tooling}
Right before jumping into the development, the roles of the team members were chosen, the available resources were considered and our availability was analyzed. We discussed how individuals can contribute to the final outcome of the project, how we can use our tutors' resources, what the main milestones in the project are and how much time can the team members dedicate to this project. 

The Gantt chart on Figure \ref{gantt-chart} summarizes the project's timeline, with the main milestones and activities. The purple lines mark the scheduled time for the activity, while the orange lines mark the time frame in which an activity was executed, but not originally planned.
 
\begin{figure}[h] 
		\begin{center}
			\includegraphics[width=1\textwidth]{images/gantt-chart.png}
			\caption{The Gantt chart of the project.}
			\label{gantt-chart}
		\end{center}
	\end{figure} 
 
% github as a working tool
To make the communication and knowledge sharing among team members, a Facebook group and a Google Drive folder was configured. For future source code version tracking, a Github page\footnote{\url{https://github.com/orgs/InteractiveOfficeProject/}} was configured. The full source code of the project and the report is made available through this page for the public. 

For the wireframes and prototype design, Microsoft Visio was used. To design the APIs, a Swagger \footnote{\url{http://swagger.io}} page was modified. Xamarin IDE \footnote{\url{https://www.xamarin.com/}} (Integrated Developer Environment) was used for the client's development.

\subsection{Chosen technologies}
The application is separated into a client and a server application.
% what resources are utilized for carrying the project out? 

% what technologies, development principles are chosen and followed and why? 

%what do we build upon?

\paragraph{Client} It was decided to develop the client in GTK\#. We chose this framework because it is platform-independent -- our client will be able to run on Linux, MacOS, and Windows -- and we were familiar with C\#. We decided to create a minimum viable product (MVP) and extend this application in small steps. 

A screenshot of the MVP is displayed in Figure \ref{fig:mvp-screenshot}. It has only 2 functionalities: to notify the user the user 25 minutes after work has been started and to notify the user 5 minutes after a break has been started. Both ``starting work'' and ``starting break'' have to be manually triggered by the users via clicking the corresponding buttons.\todo{add next feature extensions}\todo{what resources are utilized for carrying the project out?}\todo{what technologies, development principles are chosen and followed and why? }

\begin{figure}
  \centering
  \includegraphics{images/mvp-screenshot.png}
  \caption{Client MVP}
  \label{fig:mvp-screenshot}
\end{figure}

\subsection{APIs}
% what technologies were chosen and why?
The basic APIs were designed among the first components during the development. The APIs development includes the definition of the communication protocol channels, models and the possible values that are sent between the client and the server in both directions. 

In terms of technology, a traditional RESTful \footnote{url{https://en.wikipedia.org/wiki/Representational\_state\_transfer}} API was chosen and designed. The chosen data format is JSON \footnote{\url{https://en.wikipedia.org/wiki/JSON}} due to its simplicity, easiness and wide usage in the present time. 

The APIs are not discussed further and are out of the scope of this document. Nevertheless, the history of the git repository and the model's source code is available via the GitHub page \footnote{url{https://github.com/InteractiveOfficeProject/api-documentation}} and the HTML documentation is made public via the department's websites \footnote{url{http://pivanics.users.cs.helsinki.fi/interactive-office-api-documentation/}}. 