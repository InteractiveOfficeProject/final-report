\documentclass[english]{tktltiki}
\usepackage[pdftex]{graphicx}
\usepackage{subfigure}
\usepackage{booktabs}
\usepackage{url}
\usepackage{amsthm,amssymb}
 \usepackage{amsmath}
\begin{document}
\onehalfspacing

\title{Interactive Office Project}
\author{Jan Lippert, Luca Vitrini, Michael Morasch, P�ter Ivanics}
\date{\today}

\maketitle

\numberofpagesinformation{\numberofpages\ pages + \numberofappendixpages\ appendices}
\mytableofcontents

\section{Introduction}
% introductory summary to the context

% what is this report about? under what circumstances the project is carried out? 
This report is a summary for the Interactive Office project carried out in the Designing Interactive Systems course at the Department of Computer Science at University of Helsinki during the Spring term of 2017. The report introduces the project ideas, scope, presents its timeline, findings and results.

% what is the goal of this document?
The goal of this document to summarize the utilized methods, identified problems, describe the challenges and findings encountered by the researchers during the project. To make the report easier to read and understand, the chapters to follow are in chronological order similarly to the flow of the project.

\section{Topic selection and research area}
% the context in which the course was started
During the course lectures, several areas of Human-Computer Interaction (HCI) and related areas were presented and discussed. The course material covered several aspects of research in this area, such as user research, needfinding, prototyping methods and product evaluation. Furthermore, popular areas of research, such as augmented and virtual reality, ubiquitous computing, physiological computing were discussed. 

% what were the original ideas and how the interactive office came to surface?
The topic of Smart Home applications was one of the possible research to dive into, which served as the origin of the present research. Despite the fact that the technology in such application field is studied for a long time already, the solutions are yet limited to the home/living environment. By living environment we mean a setting, where people live either on their own or in a family, perform household-related activities (e.g. cooking, cleaning), spend their free time together and so on. At the same time, Smart Home solutions bring running a household to a new level by introducing interactive, computer-facilitated solutions to enhance energy-efficiency and well-being of families. 

% how does our idea differ, what are the main similarities and differences? 
At the first place our group started thinking, how such solutions could be utilized in a different setting, such as an office or a working environment. Starting from existing Smart Home solutions we could derive several aspects that could be brought into a Smart Office solution, such as: 

\begin{itemize}
	\item energy-efficiency, 
	\item automation of recurring tasks,
	\item well-being of employees in the office,
	\item enhancement of work culture and atmosphere, 
	\item workforce management, workload tracking and optimization,
	\item efficient maintenance of electronic devices and facilities.
\end{itemize}

% why is this research area important? 
Undoubtedly, the above aspects are relevant in numerous places and setting. For example, companies that operate in any field of industry are likely to have an office, where the employees do their everyday job. Naturally, it is in the interest of the corporation to address the aspects listed above to enhance efficiency, quality work and atmosphere. 

\subsection{Needfinding}
% what is the problem? 

% who are the stakeholders? 

% what are the connections between the stakeholders? 

% interview/references? what people in such environment say? 

% what are the similar challenges in a different environment? 

\section{Discussion}
\subsection{Research methods}
% what research methods are chosen

\subsection{Resources}
% what resources are utilized for carrying the project out? 

% what technologies, development principles are chosen and followed and why? 

%what do we build upon?

\subsection{Proposition}
% what do we propose? 

% what are the scenarios in which the solution can work? 

\subsection{Results and evaluation}

\section{Conclusions and future plans}

\pagebreak
\nocite{*}
\bibliographystyle{tktl}
\bibliography{references}

\lastpage
\appendices
\pagestyle{empty}
\end{document}